\begin{figure}[t!]
\begin{overpic} 
[width=\linewidth]
% [width=\linewidth,grid,tics=5]
{fig/convsurf/item.png}
% {fig/fitting/convolution_blocks}
\put(25,57){\small{$\skeleton$}}
\put(74,57){\small{$\skeleton$}}
\put(9,54){\small{$c_1$}}
\put(10,60){\small{$r_1$}}
\put(40,52.5){\small{$c_2$}}
\put(41,58){\small{$r_2$}}
\end{overpic}
% \vspace{-.3in} 
\caption{
% 
% Image thanks to: ryoichi_sig13
The \todo{sphere-mesh skeleton $\skeleton$ identifies sphere positions and radii.}
% 
\todo{The surface of the object is obtained as the convex-hull of the spheres on the vertices of the skeleton.}
% zero-crossing of this implicit function describes the convex-hull of our spheres.
% 
\todo{Sphere-meshes} can be rendered through GPU ray-tracing, or by meshing \todo{the zero-crossing of their implicit function; see \Eq{convsurf}.}
% 
% 
}
\label{fig:convsurf}
\end{figure}

\providecommand{\tangent}{\mathbf{t}}
\providecommand{\normal}{\mathbf{n}}
\begin{figure}[t]
\centering
\begin{overpic} 
[width=\linewidth]
% [width=\linewidth,grid,tics=5]
{hmodel/fig/corresp/item.png}
%--- LEFT
\put(13,12){$\ballcenter_1$}
\put(10,27){$\tangent_1$}
\put(41,11){$\ballcenter_2$}
\put(42,23){$\tangent_2$}
\put(27,12){$\skewproj$}
\put(27,21){$\footpoint$}
\put(27,28){$\normal$}
\put(28,34){$\pointHmodel$}
%--- RIGHT
\put(60,10){$\ballcenter_1$}
\put(83,05){$\ballcenter_2$}
\put(92,17){$\ballcenter_3$}
\put(59,19){$\tangent_1$}
\put(93,24){$\tangent_2$}
\put(83,13){$\tangent_3$}
\put(74,13){$\skewproj$}
\put(74,19){$\footpoint$}
\put(74,24){$\normal$}
\put(74.5,31){$\pointHmodel$}
\end{overpic}
\caption{
% 
% 
The computation of closest point correspondences on pill (left) and wedge (right) elements can be performed by tracing a ray along the normal \todo{of the line (resp. plane) tangent to the circles (resp. spheres).}
% of the circles (resp. sphere) tangent line (resp. plane).
%
%
}
\label{fig:corresp}
\end{figure}
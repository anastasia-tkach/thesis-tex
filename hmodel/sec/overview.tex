% !TEX root = ../hmodel.tex

\providecommand{\tangent}{\mathbf{t}}
\providecommand{\normal}{\mathbf{n}}
\begin{figure}[t]
\centering
\begin{overpic} 
[width=\linewidth]
% [width=\linewidth,grid,tics=5]
{hmodel/fig/corresp/item.png}
%--- LEFT
\put(13,12){$\ballcenter_1$}
\put(10,27){$\tangent_1$}
\put(41,11){$\ballcenter_2$}
\put(42,23){$\tangent_2$}
\put(27,12){$\skewproj$}
\put(27,21){$\footpoint$}
\put(27,28){$\normal$}
\put(28,34){$\pointHmodel$}
%--- RIGHT
\put(60,10){$\ballcenter_1$}
\put(83,05){$\ballcenter_2$}
\put(92,17){$\ballcenter_3$}
\put(59,19){$\tangent_1$}
\put(93,24){$\tangent_2$}
\put(83,13){$\tangent_3$}
\put(74,13){$\skewproj$}
\put(74,19){$\footpoint$}
\put(74,24){$\normal$}
\put(74.5,31){$\pointHmodel$}
\end{overpic}
\caption{
% 
% 
The computation of closest point correspondences on pill (left) and wedge (right) elements can be performed by tracing a ray along the normal \todo{of the line (resp. plane) tangent to the circles (resp. spheres).}
% of the circles (resp. sphere) tangent line (resp. plane).
%
%
}
\label{fig:corresp}
\end{figure}
\section{Tracking}
\label{sec:tracking}
Given a calibrated hand model $\model$, our real-time tracking algorithm optimizes the 28 degrees of freedom $\parpose$ (i.e. joint angles) so that our hand model matches the sensor input data; the generation of a calibrated model $\model$ for a user is detailed in \Section{modeling}. Directly extending the open source \emph{htrack} framework of \cite{tagliasacchi2015robust}, we write our tracking optimization in Gauss-Newton/Levenberg-Marquardt form:
% 
\begin{eqnarray}
\parpose_t = \argmin_{\parpose}
\sum_{\mathcal{T} \in \termstrack} 
w_\mathcal{T} E_\mathcal{T}(\depth_t,\parpose,\parpose_{t-1})
\label{eq:htrack}
\end{eqnarray}
% 
where fitting energies are combined with a number of priors to regularize the solution and ensure the estimation of plausible poses. 

The energy terms $\termstrack$ in our optimization are:
% 
\begin{description}[labelsep=0em,labelwidth=.6in,labelindent=.25cm,itemsep=-.6em]
    \item[d2m]          each data point is explained by the model
    \item[m2d]          the model lies in the sensor visual-hull
    \item[pose]         hand poses sample a low-dimensional manifold
    \item[limits]       joint limits must be respected
    \item[collision]    fingers cannot interpenetrate
    \item[temporal]     the hand is moving smoothly in time
\end{description}
% 
We limit our discussion to the computational elements that need to be adapted to support sphere-meshes, while referring the reader to \cite{tagliasacchi2015robust} for other details.

\subsection*{Hausdorff distance} 
% Generally speaking,
\todo{The similarity of two geometric models can be measured }by the symmetric Hausdorff distance $\metric_{X \leftrightarrow Y}$:
% 
\begin{eqnarray*}
\metric_{X \rightarrow Y} =& \max_{x \in X} \left[ \min_{y \in Y} \metric(x,y) \right] \\
\metric_{Y \rightarrow X} =& \max_{y \in Y} \left[ \min_{x \in X} \metric(x,y) \right] \\
\metric_{X \leftrightarrow Y} =& \max \{ d_{X \rightarrow Y}, \metric_{Y \rightarrow X} \}
\end{eqnarray*}
We therefore interpret our terms $E_{d2m}$ and $E_{m2d}$ as approximations to the asymmetric Hausdorff distances $\metric_{X \rightarrow Y}$ and $\metric_{Y \rightarrow X}$, where the difficult to differentiate \emph{max} operators are replaced by arithmetic means, and a robust $\ell_1$ distance is used~\cite{regcourse}.
% for $d(x,y)$.

\begin{figure}[t!]
\centering
\begin{overpic} 
[width=\linewidth]
% [width=\linewidth,grid,tics=10]
{hmodel/fig/posing/item.pdf}
% \put(10,10){\todo{\Large Overlay Text}}
\end{overpic}
\caption{
% 
% 
(a) A visualization of the posed kinematic frames ${\bar{T}}_*$.
(b) The kinematic chain and number of degrees of freedom for posing our tracking model.
Tracking quality with (c) optimal and (d) non-optimal kinematic transformation frames. 
% 
% 
}
\label{fig:posing}
\end{figure}

\subsection*{Data $\rightarrow$ Model}
The first asymmetric distance minimizes the average closest point projection of each point $\pointHmodel$ in the \todo{depth} frame $\depth$:
%
\begin{equation}
E_{d2m} = |\depth|^{-1} \sum_{\pointHmodel \in \depth} \| \pointHmodel - \proj_{\model(\parsHmodel)}(\pointHmodel)\|_2^1
\label{eq:d2m}
\end{equation}
% 
Adapting this energy, as well as its derivatives, to sphere-meshes requires the specification of the projection operator $\proj_\model$ that is described in \Section{corresp}.

\subsection*{Model $\rightarrow$ Data}
The second asymmetric distance considers how our monocular acquisition system does not have a complete view of the model. While the 3D location is unknown, we can penalize the model from lying outside the sensor's \emph{visual hull}:
\begin{equation}
E_{m2d} = |\model(\parsHmodel)|^{-1} \int_{\pixelHmodel \in \model(\parsHmodel)} \| \pixelHmodel - \proj_{\depth}(\pixelHmodel)\|_2^1
\label{eq:m2d}
\end{equation}
In the equation above, \todo{the integral is discretized as a sum over the set of pixels obtained through rasterization; see ~\Section{rendering}. The rasterization renders the model to the image plane using the intrinsic and extrinsic parameters of the sensor's depth camera.}

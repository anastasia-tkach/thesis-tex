% !TEX root = ../hmodel.tex

This chapter is based on the following publication:

\begin{adjustwidth}{2.5em}{0pt}
\textsc{Tkach A., Pauly M., Tagliasacchi A.}: Sphere-meshes for real-time hand
modeling and tracking. \textit{In ACM Trans. Graph. (Proc. SIGGRAPH Asia) (2016)}.
\end{adjustwidth}

\section*{Abstract}

%--- Background
Modern systems for real-time hand tracking rely on a combination of discriminative and generative approaches to robustly recover hand poses. Generative approaches require the specification of a geometric model.
%--- Core point
In this paper, \todo{we propose a the use of sphere-meshes} as a novel geometric representation for real-time generative hand tracking. 
% 
\todo{How tightly this model fits a specific user heavily affects tracking precision.}
%--- Model Adaptation
We derive an optimization to non-rigidly deform a template model to fit the user data in a number of poses.
% 
%--- Performance
This optimization jointly captures the user's static and dynamic hand geometry, thus facilitating high-precision registration.
At the same time, the limited number of primitives in the \todo{tracking template} allows us to retain excellent \todo{computational} performance. We confirm this by embedding our models in an open source real-time registration algorithm to obtain a tracker steadily running at 60Hz.
%
%--- Why should I believe you?
We demonstrate the effectiveness of our solution by qualitatively and quantitatively evaluating tracking precision on a variety of complex motions. We show that the improved tracking accuracy at high frame-rate enables stable tracking of extended and complex motion sequences without the need for \todo{per-frame} re-initialization.
%
%--- Data release
To enable further research in the area of high-precision hand tracking, \todo{we publicly release source code and evaluation datasets.}
% together with the corresponding evaluation metrics.

\begin{figure}[t]
\centering
\begin{overpic} 
[width=\linewidth]
% [width=\linewidth,grid,tics=5]
% {fig/wristband/composite.pdf}
{htrack/fig/wristband/composite.png}
% \put(0,0){\small{text}}
\put(75,02){\small{sensor cloud}$\PointsSensorHtrack$}
\put(75,22){\small{sensor silh.} $\SilhoSensor$}
\put(1,02){\small{wristband mask}}
\put(1,22){\small{depth image}}
\put(30,02){\small{PCA wristband}}
\put(53,02){\small{hand ROI}}
\put(35.5,17){$\ell$}
\putfilename
\end{overpic}
\vspace{1em}
\caption{
% \todo{[wristband]}
%
We first identify the wristband mask by color segmentation, then compute the 3D orientation of the forearm as the PCA axis of points in its proximity.
Offsetting a 3D sphere  from the wristband center allows isolating the region of interest. The obtained silhouette image and sensor point clouds are shown on the right. 
\vspace{-.2in}
% \TODO{display padding sensor silhouette}
% 
}
\label{fig:wristband}
\end{figure}

\begin{figure}[t]
\centering
\begin{overpic} 
[width=\linewidth]
% [width=\linewidth,grid,tics=5]
{htrack/fig/pcafail/composite.pdf}
\renewcommand{\yoff}{-4}
\put(05,\yoff){\small$89\%$, \#PCA=4}
\put(30,\yoff){\small$96\%$, \#PCA=6}
\put(54,\yoff){\small$99\%$, \#PCA=9}
\put(79,\yoff){\small$79\%$, \#PCA=2}
\renewcommand{\yoff}{-8}
\put(08,\yoff){\small$\omegapcaproj=10^8$}
\put(33,\yoff){\small$\omegapcaproj=10^8$}
\put(57,\yoff){\small$\omegapcaproj=10^8$}
\put(82,\yoff){\small$\omegapcaproj=10^2$}
\putfilename
\end{overpic}
\vspace{2em} % leave enough space for overpic
\caption{
%
Optimizing directly in the PCA subspace~\protect\cite{schroder2014real} can lead to inferior registration accuracy.
We replicate this behavior by setting $\omegapcaproj$ in Equation~\ref{eq:pcaproj} to a large value. Even when increasing the number of PCA bases to cover $99\%$ of the variance in the database, the model remains too stiff to conform well to the input. Our approach  is able to recover the correct hand pose by optimizing for projection distances even with a very limited number of bases (right).}
\label{fig:pcafail}
\end{figure}

% by optimizing for projection distance (\Equation{pcaproj}) we can
% 
% We directly compare our method to the one in  by varying the number of PCA bases. Note how are technique is capable to fit well to the data with a very limited number of PCA bases.
% 
% (bottom) To decouple improvements due to our fitting energies, we realize the subspace optimization of \protect\cite{schroeder_icra14} by setting the latent projection weight $\omegapcaproj=10^{8}$.
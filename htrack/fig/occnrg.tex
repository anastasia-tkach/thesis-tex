\begin{figure}[t]
\centering
\flushleft
\begin{overpic} 
[width=.97\linewidth]
% [width=.97\linewidth,grid,tics=5]
{htrack/fig/occnrg/composite.pdf}
\put(100,2){\rotatebox{90}{\small{corresp. culling}}}
\put(100,24){\rotatebox{90}{\small{occlusion energy}}}
\putfilename
\end{overpic}
\vspace{1em}
\caption{
% 
Correspondence computations.
The top row shows the strategy 
% originally proposed for full-body tracking by \protect\cite{ganapathi_eccv12}
used in~\protect\cite{qian2014realtime} adapted to our gradient-based framework according to the formulation given in~\protect\cite{wei_siga12}. The bottom row shows the improved accuracy of our new approach.
% 
%The effect of replacing our view-dependent correspondence computation by an energy minimizing depth disparity~\protect\cite{wei_siga12} for portions of the model occluding the sensor data~\protect\cite{qian_cvpr14}. \MP{I don't understand the references. So the top row is Wei et al and the bottom row our method? Why refer to Qian?}
} % caption
\label{fig:occnrg}
\end{figure}
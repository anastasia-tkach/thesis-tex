\begin{figure}[t]
\centering
\begin{overpic} 
[width=.8\linewidth]
% [width=.8\linewidth,grid,tics=5]
{htrack/fig/shapespaceproj/composite.pdf}
\put(36,45){$\parsHtrack$}
\put(26,31){$\tilde\parsHtrack$}
\put(26,23){$\boldsymbol{\mu}$}
\put(39,6){$\posespace$}
% \put(25,31){$\tilde\pars$}
\putfilename
\end{overpic}
\caption{
% 
An illustration of the energies involved in our pose-space prior. For illustration purposes the full dimensional parameter vector $\parsHtrack\in\mathbb{R}^3$, while latent space variable $\tilde\parsHtrack\in\mathbb{R}^2$.
% 
The PCA optimization in \protect\cite{schroder2014real} constrains the pose parameters $\parsHtrack$ to lie on the subspace~$\posespace$. Conversely, we penalize the distance of our pose from~$\posespace$~(\Equation{pcaproj}); simultaneously, we ensure our pose remains likely by preventing it from diverging from the mean of the distribution~(\Equation{pcareg}).
% 
% either limiting the freedom of motion of the hand.
% This reduces the freedom of motion, as only poses in the subspace
% This can result in erroneously recovered poses, as $\pars$ can
% 
}
\label{fig:shapespaceproj}
\end{figure}
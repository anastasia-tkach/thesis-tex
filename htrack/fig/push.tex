\begin{figure}[t]
\centering
\begin{overpic} 
[width=\linewidth]
% [width=\linewidth,grid,tics=5]
% {fig/push/composite.pdf} %f'd up transparency
{htrack/fig/push/composite.png}
\put(44,40){\small{$\SilhoSensor$}}
\put(44,18){\small{$\SilhoSensor$}}
\put(31.5,-2){\small{silhouette}}
\put(54.5,-2){\small{w/o silhouette}}
\put(79.5,-2){\small{w/ silhouette}}
\putfilename
\end{overpic}
\vspace{1em}
\caption{
% \todo{[push]}
Our 2D silhouette registration energy is essential to avoid tracking errors for occluded parts of the hand.
When no depth data is available for certain parts of the model, a plausible pose is inferred by ensuring that the model is contained within the sensor silhouette image $\SilhoSensor$.
% correctly track occluded geometry. In these scenarios, it is the lack of data of depth measurements in certain areas that drives pose inference - by ensuring our model is behind by the sensor silhouette $\SilhoSensor$.
}
\label{fig:push}
\end{figure}
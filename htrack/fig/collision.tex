\begin{figure}[t]
\centering
\begin{overpic} 
[width=.8\linewidth]
% [width=\linewidth,grid,tics=5]
{htrack/fig/collision.pdf}
% left
\put(10.5,14){\small $\mathbf{n}_i$}
\put(17,11.5){\small $\pointHtrack_i$}
% right
\put(30.5,14){\small $\mathbf{n}_j$}
\put(23,11.5){\small $\pointHtrack_j$}
% center
% \put(21,15){\small $\mathbf{h}$}
% right hand side figure
\put(73.5,13.5){\small{$\pointHtrack_i,\pointHtrack_j$}}
\putfilename
\end{overpic}
\caption{(left) Collision constraints definition, deepest penetration points marked as $\pointHtrack_i,\pointHtrack_j$. (right) When the collision energy is minimized in isolation the penetration points are co-located.}
\label{fig:collision}
\end{figure}
\begin{figure}[t]
\centering
\begin{overpic} 
[width=\linewidth]
% [width=\linewidth,grid,tics=5]
{htrack/fig/occlusion/composite.pdf}
\put(22,-1){\small(a)}
\put(53,-1){\small(b)}
\put(85,-1){\small(c)}
\put(18,14){$c_1$}
\put(49,14){$c_1$}
\put(80,14){$c_1$}
\put(32,6){$c_2$}
\put(63,14){$c_2$}
\put(94,14){$c_2$}
\putfilename
\end{overpic}
\vspace{1em}
\caption{
% We illustrate the necessity to compute correspondences against
% 
Illustration of the impact of self-occlusion in correspondences computations. 
(a) The finger $c_2$ initially occluded by finger $c_1$ becomes visible, which causes new samples to appear. (b)  Closest correspondences to the portion of the model visible from the camera do not generate any constraints that pull $c_2$ toward its data samples. This is the approach in \protect\cite{wei_siga12}, where these erroneous matches are then simply pruned. (c) Our method also considers front-facing portions of the model that are occluded, allowing the geometry to correctly register.      
%
}
\label{fig:occlusion}
\end{figure}
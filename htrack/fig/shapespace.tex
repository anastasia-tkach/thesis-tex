\begin{figure}[t]
\centering
\begin{overpic} 
[width=\linewidth]
% [width=\linewidth,grid,tics=5]
{htrack/fig/shapespace/composite.pdf}
\putfilename
\end{overpic}
\caption{An illustration of the PCA pose-space used to regularize the optimization. Black dots denote the samples of the data base. 
High likelihood poses are located nearby the mean of the latent space (dark red). 
% Note how the eigenvalues of the PCA space skew the metric in such a way that certain axis offer a broader spectrum of likely poses.
The eigenvalues of the PCA define the metric in the low-dimensional space, skewing it in certain directions. Poses that, according to this metric, are far from the mean are likely to be unnatural and will be penalized in the optimization.
}
\label{fig:shapespace}
\end{figure}
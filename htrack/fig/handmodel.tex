\begin{figure}[t]
\centering
\begin{overpic} 
[width=\linewidth]
% [width=\linewidth,grid,tics=5]
% {fig/handmodel/composite.png}
{htrack/fig/handmodel/composite.pdf}
\put(36,1){\tiny $\skelpoint_0$}
\put(38,9){\tiny $\skelpoint_1$}
\put(40,14){\tiny $\skelpoint_2$}
\put(42.5,19.5){\tiny $\skelpoint_3$}
\put(43.5,23){\tiny $\skelpoint_4$}
% Too cluttered if we put more!!!
\putfilename
\end{overpic}
\caption{ 
%
A visualization of the template hand model with the number and location of degrees of freedom of our optimization. From left to right: The cylinder model used for tracking, the skeleton, the BVH skeleton exported to Maya to drive the rendering, the rendered hand model. 
% Note the rig bones of fingertips are sligthly longer, as they extend until they touch the surface.
% \MP{Not sure if I understand the last sentence. Should we skip it?}
%\TODO{Replace (d) with posed rendered model.}
} % caption
\label{fig:handmodel}
\end{figure}

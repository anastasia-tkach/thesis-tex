\begin{figure}[t]
\flushleft
\begin{overpic} 
[width=.97\linewidth]
% [width=.97\linewidth,grid,tics=5]
{htrack/fig/melax/composite.pdf}
\put(101,22){\small\rotatebox{90}{Melax~\protect\cite{melax2013dynamics}}}
\put(101,7){\small\rotatebox{90}{[our]}}
\put(3.5,-2){\small{Frame 448}}
\put(19.5,-2){\small{Frame 1151}}
\put(38,-2){\small{Frame 1595}}
\put(58.8,-2){\small{Frame 1615}}
\put(80,-2){\small{Frame 1756}} % really 756
\putfilename
\end{overpic}
\vspace{1.5em}
\caption{Comparison to the method of~\protect\cite{melax2013dynamics}. The full sequence can be seen in the accompanying video. We highlight a few frames that are not resolved correctly by this method, but that can be handled successfully with our solution. The last frame shows the better geometric approximation quality of the convex body model used in \protect\cite{melax2013dynamics} compared to our simpler cylinder model.
} % caption
\label{fig:melax}
\end{figure}
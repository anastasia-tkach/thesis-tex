% !TEX root = ../htrack.tex
\begin{figure*}[t]
\centering
\begin{overpic} 
[width=\linewidth]
% [width=\linewidth,grid,tics=5]
{htrack/fig/overview/composite.pdf}
\put(92.5,19.3){\small{$\parsHtrack(t)$}}
%\put(83.5,18.6){\small{$\delta(t-2)$}}
%\put(83.5,20.25){\small{$\delta(t-1)$}}
\put(63.0,4){\small{$[\underline\parsHtrack,\overline\parsHtrack]$}}
\putfilename
\end{overpic}
\caption{
% 
Overview of our algorithm. For each acquired frame we extract a 3D point cloud of the hand and the 2D distance transform of its silhouette. 
% 
From these we compute point correspondences to align a cylinder model of the hand to best match the data. This registration is performed in an ICP-like optimization that incorporates a number of regularizing priors to ensure accurate and robust tracking.
%fit our hand model to the data. In an ICP fashion, these correspondences are computed using the cylinder model posed and the silhouette image of from the previous solver iteration.
% 
%A number of priors regularize tracking, amongst which the result of previously tracked frames. Our solver optimizes for both joint angles and global transformation of a skeletal hand model.}
}
\label{fig:overview}
\end{figure*}

\section{Related Work}
\label{sec:htrack-related}



We discuss the most relevant papers on human hand tracking related to our approach. For a more general overview of human motion analysis using depth sensors we refer to the recent survey of~\cite{ye_13}.

Tracking algorithms can be roughly divided into two main classes, \emph{appearance-based} and \emph{model-based} methods~\cite{erol2007vision}.
Appearance-based approaches train a classifier or a regressor to map image features to hand poses. Consequently, while these systems can robustly determine a hand pose from a single frame, appearance-based methods are optimal in scenarios where only a rough pose estimate is desired~\cite{wang_uist11} or highly discriminative features can be extracted~\cite{wang2009colorglove}. Conversely, \emph{model-based} techniques approach tracking as an alignment optimization, where the objective function typically measures the discrepancy between the data synthesized from the model and the one observed by the sensor. While model-based methods can suffer from loss-of-tracking~\cite{wei_siga12}, regularizing priors can be employed to infer high-quality tracking even when sensor data is incomplete or corrupted~\cite{melax2013dynamics,schroder2014real}. 
In this work we focus on improving the robustness and accuracy of model-based approaches by combining effective 2D and 3D registration energies with carefully designed priors.

\subsection*{Appearance-based hand tracking} 
During the past few years, numerous appearance-based methods have been developed for hand tracking. Approaches based on nearest neighbor search~\cite{wang2009colorglove,wang_uist11,romero_13}, decision trees~\cite{keskin2012hand,tang2013real,tang_cvpr14,krupka2014discriminative}, or convolutional networks \cite{tompson2014real} have demonstrated that appearance-based methods can be successfully employed for realtime hand tracking. The strength of these methods is the capability of inferring a hand pose from a single frame without the need of relying on temporal coherence, which avoids drift.
 However, such appearance-based approaches are tightly linked to the training data and often do not generalize well to previously unseen hand poses, i.e., poses not contained in the training database. For this reason most of these methods assume a single hand in isolation to avoid data explosion, and often do not reach the accuracy of model-based methods.

\subsection*{Model-based hand tracking}

A popular approach to hand motion capture is to use a marker-based system (e.g.\ Vicon, OptiTrack).  A 3D hand model can then be fitted to the tracked markers to get the final hand poses. A small number of markers has been shown to be sufficient for reconstructing the 3D hand poses via inverse kinematics techniques~\cite{Hoyet_i3d12}. However, due to frequent occlusions of the markers, motion sequences acquired using marker-based systems often need a significant amount of manual cleaning. To overcome this issue, \cite{zhao2012marker} propose to complement a marker-based system with RGBD data to capture hand motion even in case of significant self-occlusion.
Recently, accurate model-based tracking has been achieved in a multiple camera setup~\cite{sridhar2013multicam,sridhar2014anisotropic}, where the multiple vantage points help resolving challenging occlusions. Multiple camera systems have also been used successfully to model precise hand-hand and hand-object interactions~\cite{oiko_iccv11,ballan2013salient,wang2013physics}. All of the above methods require a complex acquisition setup and manual calibration, which makes them less suitable for the kind of consumer-level applications that we target with this work.

\emph{Particle-swarm optimization} (PSO) methods achieve interactive (15~fps) tracking with a single RGBD camera~\cite{oiko2011hand}. PSO techniques have also been applied successfully to model challenging interaction between two hands \cite{oiko_cvpr12} at a reduced rate of 4 fps. 
PSO is an optimization heuristic that does not use the gradient information of the considered optimization problem, but instead uses a sampling strategy.
For this reason the accuracy and efficiency of PSO approaches heavily rely on the number of samples used. Oikonomidis et al.~\cite{oikonomidis2014evolutionary} introduced a more advanced sampling strategy that improves tracking efficiency without compromising quality. However, gradient-based optimization approaches  converge faster and more accurately than PSO when close to the solution, and are therefore well suited for realtime applications~\cite{qian2014realtime}.


Compelling 60 fps realtime performance was recently shown using gradient-based optimization by~\cite{melax2013dynamics}, where the optimization is expressed as a convex rigid body simulation, and numerous heuristics for re-initialization were employed to avoid tracking failures. Rather than resorting to reinitialization for robustness, \cite{schroder2014real} formulate the optimization in a subspace of likely hand poses. While the lower number of optimization variables leads to efficient computations, tracking accuracy can be limited by the reduced pose complexity induced by the subspace.

 In this paper, we show that hand tracking can be formulated as a single gradient-based optimization to obtain an efficient and accurate real-time tracking system running at up to 120 fps. By using a combination of geometric and data-driven priors we achieve significant improvements in tracking quality and robustness.

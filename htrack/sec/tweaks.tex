% Comments
\definecolor{turquoise}{cmyk}{0.65,0,0.1,0.3}
\definecolor{purple}{rgb}{0.65,0,0.65}
\definecolor{dark_green}{rgb}{0, 0.5, 0}
\definecolor{orange}{rgb}{0.8, 0.6, 0.2}
\definecolor{red}{rgb}{0.8, 0.2, 0.2}
\definecolor{blueish}{rgb}{0.0, 0.7, 1}
\definecolor{light_gray}{rgb}{0.7, 0.7, .7}
\definecolor{pink}{rgb}{1, 0, 1}
\newcommand{\edit}[1]{\textcolor{dark_green}{#1}}
\newcommand{\new}[1]{{#1}}
\newcommand{\LEGACY}[1]{\textcolor{orange}{[LEGACY] #1}}
\newcommand{\aedit}[1]{\textcolor{purple}{#1}}
\newcommand{\ADDRESSED}[1]{{}}
\newcommand{\copypaste}[1]{{#1}}
\newcommand{\revision}[1]{{#1}}
\newcommand{\hidden}[1]{{}}

%%
%% inlined comments
%% 
\newcommand{\MP}[1]{{\color{orange}[MP: #1]}} % mark 
\newcommand{\AT}[1]{{\color{blueish}[AT: #1]}} % andrea
\newcommand{\SB}[1]{{\color{turquoise}[SB: #1]}} % sofien
\newcommand{\NT}[1]{{\color{dark_green}[NT: #1]}} % anastasia
\newcommand{\MB}[1]{{\color{pink}[MB: #1]}} % mario
\newcommand{\MS}[1]{{\color{purple}[MS: #1]}} % matthias

\newcommand{\yoff}{0}

%%
%% shortcut for references
%% 
\newcommand{\Fig}[1]{Figure~\ref{fig:#1}}
\newcommand{\Figure}[1]{Figure~\ref{fig:#1}}
\newcommand{\Eq}[1]{Equation~\ref{eq:#1}}
\newcommand{\Equation}[1]{Equation~\ref{eq:#1}}
\newcommand{\Sec}[1]{Section~\ref{sec:#1}}
\newcommand{\Section}[1]{Section~\ref{sec:#1}}
\newcommand{\Appendix}[1]{Appendix~\ref{app:#1}}


%%
%% Inlined annotation of paragraph content
%% 
\newcommand{\brief}[1]{} % ONLY LATEX VISIBLE
% \newcommand{\brief}[1]{{\flushright\small{\vspace{-7pt}\color{light_gray} [\textbf{#1}]}\\}}


%%
%% MATH STUFF
%% 
\let\vec=\mathbf
\let\mat=\vec
\newcommand{\F}{F} % dataframe
\newcommand{\normal}{\mathcal{N}}
\newcommand{\vertex}{\mathbf{v}}

% MARIO
\newcommand{\abs}[1]{\left| #1 \right|}
\newcommand{\norm}[1]{\left\Vert {#1} \right\Vert}

% hack to use computer modern font for bold greek letters
\SetSymbolFont{letters}{bold}{OML}{cmm}{b}{it}
\SetSymbolFont{operators}{bold}{OT1}{cmr}{bx}{n}

%%
%% Small math symbols
%%
\newcommand{\unaryminus}{\scalebox{0.5}[1.0]{\( - \)}} % small "-"
% \newcommand{\smallless}{\scalebox{0.5}[.5]{\(<\)}} % small "<"

%--- OLD SYMBOLS
% \newcommand{\pars}{\boldsymbol{\omega}}
% \newcommand{\parsopt}{\boldsymbol{\omega}^*}

%%
%% Inverted
%%
\usepackage[T3,T1]{fontenc}
\DeclareSymbolFont{tipa}{T3}{cmr}{m}{n}
\DeclareMathAccent{\invbreve}{\mathalpha}{tipa}{16}

%%
%% Math symbols
%%
\newcommand{\cyl}{\mathbf{c}} %cylinder
\newcommand{\cylrad}{\mathbf{c}} %cylinder

% weights of terms
\newcommand{\omegacloud}{\omega_1}
\newcommand{\omegasilhouette}{\omega_2}
\newcommand{\omegawrist}{\omega_3}
\newcommand{\omegapcaproj}{\omega_4}
\newcommand{\omegapcareg}{\omega_5}
\newcommand{\omegacollision}{\omega_6}
\newcommand{\omegabounds}{\omega_7}
\newcommand{\omegatemporalst}{\omega_8}
\newcommand{\omegatemporalnd}{\omega_9}
%\newcommand{\skeleton}{\mathcal{K}}
\newcommand{\numpars}{26}
\newcommand{\jacobian}{\mathbf{J}}
\newcommand{\J}{\jacobian}
\newcommand{\Jproj}{\jacobian_{\text{persp}}}
\newcommand{\Jskel}{\jacobian_{\text{skel}}}
\newcommand{\residuals}{\mathbf{e}}
\newcommand{\weight}{\omega}
%\newcommand{\pixel}{\mathbf{p}}
\newcommand{\pixels}{\mathbf{P}}
%\newcommand{\point}{\mathbf{x}}
\newcommand{\skelpoint}{\mathbf{k}}
\newcommand{\points}{\mathbf{X}}
\newcommand{\proj}{\Pi}
\newcommand{\PointsRender}{\mathcal{X}_{r}}
%\newcommand{\PointsSensor}{\mathcal{X}_{s}}
\newcommand{\DepthRender}{\mathcal{D}_{r}}
\newcommand{\DepthSensor}{\mathcal{D}_{s}} 
\newcommand{\SilhoRender}{\mathcal{S}_{r}}
\newcommand{\SilhoSensor}{\mathcal{S}_{s}}
\newcommand{\DataSensor}{\mathcal{F}}
\newcommand{\history}{\mathcal{H}}
\newcommand{\handmodel}{\mathcal{M}}
\newcommand{\visiblehand}{\invbreve\handmodel}
\newcommand{\posespace}{\mathcal{P}}
\newcommand{\MU}{\boldsymbol\mu} 
\newcommand{\SIGMA}{\boldsymbol{\Sigma}}
%\newcommand{\pars}{\boldsymbol{\theta}}
\newcommand{\dpars}{\Delta\boldsymbol{\theta}}
\newcommand{\parssub}{\boldsymbol{\tilde\theta}}
\newcommand{\parsopt}{\pars^*}
\newcommand{\identity}{\mathbf{I}}
\newcommand{\zero}{\mathbf{0}}

% \underset{x}{\operatorname{argmin}} 
\DeclareMathOperator*{\argmin}{arg\,min}
\DeclareMathOperator*{\argmax}{arg\,max}


%--- 2x2 matrix block
\newcommand{\matblocktwo}[4]{
\left[
    \begin{array}{c|c}
        #1 & #2 \\ \hline
        #3 & #4 
    \end{array}
\right] 
}

%--- TO SHOW PATH OF INSERTED IMAGES
\usepackage{currfile}
% \newcommand{\putfilename}{\put(-3,0){\rotatebox{90}{\color{red}\currfilename}}}
\newcommand{\putfilename}{}

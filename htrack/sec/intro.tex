
\begin{figure}[t]
\centering
\begin{overpic} 
[width=\linewidth]
% [width=\linewidth,grid,tics=5]
{htrack/fig/data/composite.pdf}
\put(55,2){$\PointsSensorHtrack$}
\put(55,21){$\PointsSensorHtrack$}
\put(75,2){$\PointsSensorHtrack$}
\put(75,21){$\PointsSensorHtrack$}
\put(95,2){$\SilhoSensor$}
\put(95,21){$\SilhoSensor$}
\putfilename
\end{overpic}
\caption{
%
The two different sensors used in our experiments provide data with substantially different characteristics. Top: Intel's Creative Interactive Gesture camera (time of flight) provides a complete silhouette image \revision{$\SilhoSensor$}, but low quality depth measurements, resulting in severe noise in the point cloud \revision{$\PointsSensorHtrack$}. Bottom: Point clouds acquired by the PrimeSense camera (structured light) are much smoother, but the silhouette image can contain significant gaps.
\vspace{-.2in}
% 
} % caption
\label{fig:data}
\end{figure}

\section{Introduction} \label{sec:htrack-intro}

\brief{why is the problem important}
Tracking and animating humans in motion is a fundamental problem in computer graphics and computer vision. A particularly important question is how to accurately reconstruct the shape and articulation of human hands. Hand motion is a crucial component of non-verbal communication, plays an important role in the animation of humanoid avatars, and is central for numerous human-computer interfaces. 

Accurate realtime body tracking~\cite{shotton_cvpr11,wei_siga12} and face tracking~\cite{cao_sig14} systems have been recently proposed. Hand tracking is now gaining traction in the research community as a next natural step towards a complete system for online human communication in desktop environments~\cite{oiko2011hand,melax2013dynamics,sridhar2014anisotropic,schroder2014real,tompson2014real}.

Recent industrial trends in interaction systems for virtual environments have lead to the development of (closed source) software packages for the processing of RGBD data, like the Intel RealSense SDK, or purpose-designed hardware, like the Leap Motion and the Nimble sensors. 

\brief{why other approaches suck}

In this paper we introduce a system for \emph{realtime} hand tracking suitable for personal desktop environments. Our \emph{non-invasive} setup using a single commodity RGBD~sensor does not require the user to wear a glove or markers. 
Such single-camera acquisition is particularly advantageous as it is cheap, does not require any sensor calibration, and does not impede user movements. 

\brief{why is it still challenging}
Accurate hand tracking with a non-invasive sensing device in realtime is a challenging scientific problem. Human hands are highly articulated and therefore require models with sufficiently many degrees of freedom to adequately describe the corresponding motion space. Hand motion is often fast and exhibits intricate geometric configurations with complex contact patterns among fingers. 

With a single-camera RGBD setup, we are faced with incomplete data due to self-occlusions and high noise levels (see Figure~\ref{fig:data}).

Yet the simplicity of the hardware and the ease of deployment make this setup the most promising for consumer applications as evidenced by the recent proliferation of new consumer-level sensors.

To cope with the limited amount of available information, we employ an articulated template model as a geometric prior for shape completion and topology control. Our model does not only encode geometry, but also serves as a domain to represent information about \emph{plausible} hand poses and motions. This statistical information, built by analyzing a database of annotated poses, is directly embedded into the optimization, which
allows accurate tracking with a high number of degrees of freedom even in challenging scenarios.

\subsection*{Contributions}
% 
We present a complete system for realtime hand tracking using a single commodity RBGD input sensor. Our core technical contributions are:
% 
\begin{figure}[t]
\centering
\begin{overpic} 
[width=\linewidth]
% [width=\linewidth,grid,tics=5]
% {fig/handmodel/composite.png}
{htrack/fig/handmodel/composite.pdf}
\put(36,1){\small $\skelpoint_0$}
\put(38,9){\small $\skelpoint_1$}
\put(40,14){\small $\skelpoint_2$}
\put(42.5,19.5){\small $\skelpoint_3$}
\put(43.5,23){\small $\skelpoint_4$}
% Too cluttered if we put more!!!
\putfilename
\end{overpic}
\caption{ 
%
A visualization of the template hand model with the number and location of degrees of freedom of our optimization. From left to right: The cylinder model used for tracking, the skeleton, the BVH skeleton exported to Maya to drive the rendering, the rendered hand model. 
% Note the rig bones of fingertips are sligthly longer, as they extend until they touch the surface.
% \MP{Not sure if I understand the last sentence. Should we skip it?}
%\TODO{Replace (d) with posed rendered model.}
} % caption
\label{fig:handmodel}
\end{figure}


\begin {itemize}
\item \new{a novel articulated registration algorithm that efficiently integrates data and regularization priors into a \emph{unified} real-time solver; see \Section{optimization} and \Appendix{gpu},}
\item a combined 2D/3D registration method to align the 3D hand model to the acquired depth map and extracted silhouette image; see \Section{fitting},
\item a new way of computing data-to-model correspondences that accounts for occlusions and significantly improves the robustness of the tracking; see \Section{fitting},
\item a new regularization strategy that combines a statistical pose-space prior with kinematic and temporal priors to \new{simultaneously} ensure the inferred hand poses are plausible
%--- This is about the PCA-mean discussion we had in the rebuttal.
\new{and aid the algorithm in recovering from loss-of-tracking; see \Section{prior},}
\item \new{exposing an interesting relationship between the well known point-to-plane registration energy and Gauss-Newton linearization;  
see \Appendix{lindistance}.}
\end{itemize}

Another important contribution of our paper is that we fully disclose our source code\footnote{\url{https://github.com/OpenGP/htrack}}. To the best of our knowledge, no other freely available implementation is available, and we believe that publishing our code will not only ensure reproducibility of our results, but also facilitate future research in this domain.

\new{Note that there is a widespread belief \cite{wei_siga12,zhang_siga14,qian2014realtime} that ICP-like techniques are too local and prone to local minima to successfully deal with fast articulated motion. One of our contributions is to show this commonly held belief should be re-considered. We demonstrate that a regularized geometric registration approach in the spirit of ICP can achieve outstanding performance. We believe this will significantly impact future research in this domain, as it will allow further development of registration techniques for real-time tracking, in contraposition to commonly employed techniques from the vision community like  discriminative~\cite{tompson2014real} and PSO~\cite{qian2014realtime} methods.}

Our regularized geometric registration achieves robust, highly articulated hand tracking at up to 120 frames per second (fps).
%
We quantitatively and qualitatively compare the performance of our algorithm to recent appearance-based and model-based techniques (see \Section{eval}). These comparisons show a significant improvement in accuracy and robustness compared to the current state-of-the-art.

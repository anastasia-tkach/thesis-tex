% !TEX root = ../htrack.tex


\begin{figure}[t]
\flushleft
\begin{overpic} 
[width=\linewidth]
% [width=.97\linewidth,grid,tics=5]
{htrack/fig/frontcorr/composite.pdf}
\put(98, 1){\tiny\rotatebox{90}{Converged}}
\put(98,18.7){\tiny\rotatebox{90}{First Iter.}}
\put(98,34.7){\tiny\rotatebox{90}{Initial.}}
\put(15,-3){\small(a)}
\put(43,-3){\small(b)}
\put(78,-3){\small(c)}
\put(9,32){\tiny$\PointsSensor$}
\put(25,43){\tiny$\handmodel$}
\put(17,37){\color{red}$\proj_\handmodel$}
\put(45,37){\color{red}$\proj_\handmodel$}
\put(73,37){\color{red}$\proj_{\visiblehand}$}
\putfilename
\end{overpic}
% \vspace{-.05in}
\caption{
% 
Illustration of correspondences computations. The circles represent cross-sections of the fingers, the small black dots are samples of the depth map. (a) A configuration that can be handled by standard closest point correspondences.
 (b) Closest point correspondences to the back of the cylinder model can cause the registration to fall into a local minimum. Note that simply pruning correspondences with back-pointing normals would not solve this issue, as no constraints would remain to pull the finger towards the data. (c) This problem is resolved by taking visibility into account, and computing closest points only to the portion $\visiblehand$ of $\handmodel$ facing the camera. % 
}
\label{fig:frontcorr}
\end{figure}

\section{Optimization}
\label{sec:optimization}
In this section we derive the objective functions of our model-based optimization method
and provide the rationales for our design choices. 
Let $\DataSensor$ be the sensor input data consisting of a 3D point cloud $\PointsSensorHtrack$ and 2D silhouette $\SilhoSensor$ (see \Figure{data}). Given a 3D hand model $\handmodel$ with joint parameters $\parsHtrack = \{ \theta_1, \theta_2,\hdots, \theta_{26} \}$, we aim at recovering the pose $\parsHtrack$ of the user's hand, matching the sensor input data $\DataSensor$. 
To achieve this goal, we solve the optimization 
problem
% 
%\begin{equation}
%\label{eq:tracking_optimization}
%\argmin_{\pars} E_{\text{fit}} + E_{\text{prior}},
%\end{equation}
%
\begin{equation}
\label{eq:tracking_optimization}
\min_{\parsHtrack} \:\: \underbrace{E_{\text{3D}} + E_{\text{2D}} \new{+ E_{\text{wrist}}}}_{\text{Fitting terms}} + \underbrace{E_{\text{pose}} + E_{\text{kin.}} + E_{\text{temporal}}}_{\text{Prior terms}},
\end{equation}
%
combining fitting terms that measure how well the hand parameters~$\parsHtrack$ represent the data frame $\DataSensor$, with prior terms that regularize the solution to ensure realistic hand poses. For brevity of notation we omit the arguments $\parsHtrack, \PointsSensorHtrack,\SilhoSensor$ of the energy terms. We first introduce the fitting terms and present our new solution to compute tracking correspondences. Then we discuss the prior terms and highlight their benefits in terms of tracking accuracy and robustness. 
%
More details on the  implementation of the optimization algorithm will be given in \Section{implementation} and the appendix.

\begin{figure}[t]
\centering
\begin{overpic} 
[width=\linewidth]
% [width=\linewidth,grid,tics=5]
{htrack/fig/occlusion/composite.pdf}
\put(22,-1){\small(a)}
\put(53,-1){\small(b)}
\put(85,-1){\small(c)}
\put(18,14){$c_1$}
\put(49,14){$c_1$}
\put(80,14){$c_1$}
\put(32,6){$c_2$}
\put(63,14){$c_2$}
\put(94,14){$c_2$}
\putfilename
\end{overpic}
\vspace{1em}
\caption{
% We illustrate the necessity to compute correspondences against
% 
Illustration of the impact of self-occlusion in correspondences computations. 
(a) The finger $c_2$ initially occluded by finger $c_1$ becomes visible, which causes new samples to appear. (b)  Closest correspondences to the portion of the model visible from the camera do not generate any constraints that pull $c_2$ toward its data samples. This is the approach in \protect\cite{wei_siga12}, where these erroneous matches are then simply pruned. (c) Our method also considers front-facing portions of the model that are occluded, allowing the geometry to correctly register.      
%
}
\label{fig:occlusion}
\end{figure}

\subsection{Fitting Energies}
\label{sec:fitting}


%\begin{figure}[t!]
\centering
\begin{overpic} 
[width=\linewidth]
% [width=\linewidth,grid,tics=10]
{\currfiledir/item.pdf}
% Legend/1
\put(9,56){{\small $E_{2D}$ }}
\put(22,57.15){{\tiny \cite{tagliasacchi2015robust}}}
\put(22,55.15){{\tiny \citeme{}}}
% Legend/2
\put(44.5,56){{\small $E_{3D}$ }}
\put(57,57.15){{\tiny \cite{tagliasacchi2015robust}}}
\put(57,55.15){{\tiny \citeme{}}}
% Horizontal axis
\put(10.5,2){{\small \emph{tayl1} }}
\put(21.5,2){{\small \emph{srid1} }}
\put(32,2){{\small \emph{srid2} }}
\put(43,2){{\small \emph{srid3} }}
\put(54,2){{\small \emph{srid4} }}
\put(64.5,2){{\small \emph{shar1} }}
\put(75.5,2){{\small \emph{shar2} }}
\put(86.3,2){{\small \emph{shar2} }}
% Vertical axis
\put(4,11.3){{\small 1}}
\put(4,19){{\small 2}}
\put(4,26.5){{\small 3}}
\put(4,34.2){{\small 4}}
\put(4,42){{\small 5}}
\put(4,49.7){{\small 6}}
% 
\end{overpic}
\caption{
% 
% 
Average 2D/3D tracking performance metrics of the proposed method compared to \protect\cite{tagliasacchi2015robust}. 
% 
In the additional material we report error plots through time for the aggregated data above.
% 
% 
}
\label{fig:barchart}
\end{figure}
\section{Calibration}
\label{sec:modeling}

\brief{Multi-Pose Data}
Our calibration procedure adapts our template model to a specific user from a set of $N$ 3D measurements $\{ \depth_1 \dots \depth_N \}$ of the user's hand in different poses. Multiple measurements are necessary, as it is not possible to understand the kinematic behavior by analyzing static geometry, and the redundancy of information improves fitting precision. Further, in  monocular acquisition this redundancy is essential, as single-view data is highly incomplete\todo{,} making the problem ill-posed. In our research we have experimented with datasets $\{\depth_n\}$ acquired via multi-view stereo (e.g. \emph{Agisoft Photoscan}), as well as a single RGBD sensor. 
Our calibration formulation can be employed for both acquisition modalities.
\todo{Dynamic reconstruction frameworks such as~\cite{newcombe2015dynfusion} or \cite{innmann2016volume} could also be used to generate a dynamic template mesh over which sphere-mesh decimation could be executed~\cite{thiery2016spheremesh}. 
However, as no public implementation is currently available, it is currently unclear how well these methods would cope with loop-closure for features as small as human fingers.}

\paragraph{Kinematics}
The rest-pose geometry of our model is fully specified by two matrices specifying the set of sphere positions $\restcenters$ and \todo{the set of} radii $\radii$. The geometry is then posed through the application of  kinematic chain transformations; see \Figure{posing}a. Given a point $\bar\point$ on the model $\model$ at rest pose, its 3D position after posing can be computed by evaluating the expression:
% 
\begin{equation}
\point = \left[ \Pi_{k \in K(\bar\point)} \mathbf{\bar{T}}_k \mathbf{T}_k \mathbf{\bar{T}}_k^{-1} \right] \bar\point
\label{eq:kinematic}
% we apply the posing transformation, then we re-apply the rest pose transformation to bring the point back in world coordinates
\end{equation}
%
where $\mathbf{T}_*$ are the \emph{pose} transformations parameterized by $\parpose$ and $\Pi$ left multiplies matrices by recursively traversing the kinematic chain $K$ of point $\bar\point$ towards the root~\cite{buss_04}. 
Each node $k$ of the kinematic chain is associated with an orthogonal frame $\mathbf{\bar{T}}_k$ according to which local transformations are specified. In most tracking systems, the frames $\mathbf{\bar{T}}_*$ are manually set by a 3D modeling artist and kept fixed across users. However, incorrectly specified kinematic frames can be highly detrimental to tracking quality; see \Figure{posing}(c,d) and \VideoKinematic{}. Therefore, in our formulation, the kinematic structure (i.e. the matrices $\mathbf{\bar{T}}_*$) is directly optimized from acquired data.

\begin{figure}[t!]
\centering
\begin{overpic} 
[width=\linewidth]
% [width=\linewidth,grid,tics=10]
{\currfiledir/item.pdf}
% Legend/1
\put(9,56){{\small $E_{2D}$ }}
\put(22,57.15){{\tiny \cite{tagliasacchi2015robust}}}
\put(22,55.15){{\tiny \citeme{}}}
% Legend/2
\put(44.5,56){{\small $E_{3D}$ }}
\put(57,57.15){{\tiny \cite{tagliasacchi2015robust}}}
\put(57,55.15){{\tiny \citeme{}}}
% Horizontal axis
\put(10.5,2){{\small \emph{tayl1} }}
\put(21.5,2){{\small \emph{srid1} }}
\put(32,2){{\small \emph{srid2} }}
\put(43,2){{\small \emph{srid3} }}
\put(54,2){{\small \emph{srid4} }}
\put(64.5,2){{\small \emph{shar1} }}
\put(75.5,2){{\small \emph{shar2} }}
\put(86.3,2){{\small \emph{shar2} }}
% Vertical axis
\put(4,11.3){{\small 1}}
\put(4,19){{\small 2}}
\put(4,26.5){{\small 3}}
\put(4,34.2){{\small 4}}
\put(4,42){{\small 5}}
\put(4,49.7){{\small 6}}
% 
\end{overpic}
\caption{
% 
% 
Average 2D/3D tracking performance metrics of the proposed method compared to \protect\cite{tagliasacchi2015robust}. 
% 
In the additional material we report error plots through time for the aggregated data above.
% 
% 
}
\label{fig:barchart}
\end{figure}
\begin{figure}[t!]
\centering
\begin{overpic} 
[width=\linewidth]
% [width=\linewidth,grid,tics=10]
{\currfiledir/item.pdf}
% Legend/1
\put(9,56){{\small $E_{2D}$ }}
\put(22,57.15){{\tiny \cite{tagliasacchi2015robust}}}
\put(22,55.15){{\tiny \citeme{}}}
% Legend/2
\put(44.5,56){{\small $E_{3D}$ }}
\put(57,57.15){{\tiny \cite{tagliasacchi2015robust}}}
\put(57,55.15){{\tiny \citeme{}}}
% Horizontal axis
\put(10.5,2){{\small \emph{tayl1} }}
\put(21.5,2){{\small \emph{srid1} }}
\put(32,2){{\small \emph{srid2} }}
\put(43,2){{\small \emph{srid3} }}
\put(54,2){{\small \emph{srid4} }}
\put(64.5,2){{\small \emph{shar1} }}
\put(75.5,2){{\small \emph{shar2} }}
\put(86.3,2){{\small \emph{shar2} }}
% Vertical axis
\put(4,11.3){{\small 1}}
\put(4,19){{\small 2}}
\put(4,26.5){{\small 3}}
\put(4,34.2){{\small 4}}
\put(4,42){{\small 5}}
\put(4,49.7){{\small 6}}
% 
\end{overpic}
\caption{
% 
% 
Average 2D/3D tracking performance metrics of the proposed method compared to \protect\cite{tagliasacchi2015robust}. 
% 
In the additional material we report error plots through time for the aggregated data above.
% 
% 
}
\label{fig:barchart}
\end{figure}

% \newpage
\paragraph{Formulation}
Let $\parpose_n$ be the \emph{pose} parameters optimally aligning the rest-pose template to the data frame $\depth_n$, and $\parposture$ be the \emph{posture} parameters representing the transformations $\mathbf{\bar{T}}_*$ via Euler angles. 

For notational brevity, we also define $\pars_n=[\parpose_n, \parposture, \restcenters, \radii]$. Our calibration optimization can then be written as:
% 
\begin{eqnarray}
% \parposture, \centers, \radii =
\argmin_{\{\pars_n\}}
\sum_{n=1}^N 
\sum_{\mathcal{T} \in \termscalib} 
w_\mathcal{T} E_\mathcal{T}(\depth_n, \pars_n)
\label{eq:calibration}
\end{eqnarray}
% 
We employ a set of energies $\termscalib$ to account for different requirements. On one hand we want a model that is a good fit to the data; on the other, we seek a non-degenerate sphere-mesh template that has been piecewise-rigidly posed. The following calibration energies $\termscalib$ encode these requirements:
% 
\begin{description}[labelsep=0em,labelwidth=.4in,labelindent=1cm,itemsep=-.6em]
\item[d2m] data to model distance
\item[m2d] model to data distance
\item[rigid] elements are posed rigidly
\item[valid] elements should not degenerate
\end{description}
% 
To make this calibration more approachable numerically, we rewrite \Eq{calibration} as an alternating optimization problem:
% 
\begin{eqnarray}
% \restcenters, \radii, \posedcenters =
\argmin_{\posedcenters, \restcenters, \radii} &
\sum_{n=1}^N 
\sum_{\mathcal{T} \in \termscalib}
w_\mathcal{T} E_\mathcal{T}(\depth_n, \centers_n, \restcenters, \radii)
\label{eq:step1}
\\
% \parposes, \parposture =
\argmin_{\parposes,\parposture} &
\sum_{n=1}^N 
\sum_{\mathcal{T} \in \termscalib}
w_\mathcal{T} E_\mathcal{T}(\centers_n, \pars_n) 
\end{eqnarray}
% 
\todo{Our first step adjusts rest-pose sphere centers $\restcenters$ and radii $\radii$,} by allowing the model to fit to the data without any kinematic constraint beyond rigidity, and returning as a side product a set of \emph{per-frame} posed centers $\posedcenters$. 
Our second step takes the set $\posedcenters$ and projects it onto the manifold of kinematically plausible template deformations. 
This results in the optimization of the rotational components of rest-pose transformations $\mathbf{\bar{T}}_*$, as their translational components are simply derived from $\restcenters$.

\paragraph{Optimization}
The energies above are non-linear and non-convex, but can be optimized offline, as real-time tracking only necessitates a pre-calibrated model. For this reason, we conveniently employ the $lsqnonlin$ Matlab routine, which requires the gradients of our energies as well as an initialization point.
The initialization of $\restcenters$ is performed automatically by anisotropically scaling the vertices of a generic template to roughly fit the rest pose. The initial transformation frame rotations $\parposture$ are retrieved from the default template, while $\parposes$ are obtained by either aligning the scaled template to depth images, or by executing inverse kinematics on a few manually selected keypoints (multi-view stereo).
% 
% \AnastasiaComment{We should say that we use our tracking system with automatically scaled model to get the initial poses. Because the initial poses for calibration from sensor data are extremely close to the final ones. The model will not align to the from the rest pose, only tracking can do this. AT: I already wrote that, the sentence ``aligning the scaled template to the depth images''}
% 
Our (unoptimized) Matlab script calibrates the model within a few minutes for all our examples.


\subsection{Energies}
Our fitting energies are analogous to the ones used in tracking. They approximate the symmetric Hausdorff distance, but they are evaluated on \todo{a \emph{collection} of $N$ frames}:
% 
\begin{eqnarray}
E_{d2m} = 
\sum_{n=1}^N |\depth_n|^{-1} 
\sum_{\point \in \depth_n} 
\| \point - \proj_{\model(\pars_n)}(\point)\|_2^1 \\
E_{m2d} = 
\sum_{n=1}^N |\model(\pars_n)|^{-1} 
\sum_{\pixel \in \model(\pars_n)} 
\| \pixel - \proj_{\depth_n}(\pixel)\|_2^1
\end{eqnarray}
% 
Note that the projection operator $\proj_{\depth_n}$ changes according to the type of input data. If a multi-view acquisition system is used to acquire a complete point cloud, then the projection operator fetches the closest point to $\point$ in the point cloud of frame $\depth_n$. If $\depth_n$ is acquired through monocular acquisition, then $\proj_{\depth_n}$ computes the 2D projection to the image-space silhouette of the model.

\paragraph{Rigidity}
It is essential to estimate a \todo{single user template that, once articulated,} \emph{jointly} fits the set of data frames $\{ \depth_n \}$. For this purpose we require each posed model to be a piecewise-rigid articulation of our rest pose. \todo{This can be achieved by constraining each segment $\{ (\ballcenter_{n,i}, \ballcenter_{n,j})\:|\: ij \in \skeleton \}$ of $\centers_n$ to have the same length as the corresponding segment $(\bar\ballcenter_i, \bar\ballcenter_j)$ of the rest pose configuration $\restcenters$}:
% 
\begin{equation}
E_{\text{rigid}} = 
% \sum_{\ballcenter_{n,*} \in \centers_n}
\sum_{ij \in \skeleton} (\| \ballcenter_{n,i} - \ballcenter_{n,j} \| - \| \bar\ballcenter_i - \bar\ballcenter_j \|)^2
\end{equation}
% 
Note that only a subset of the edges of our control skeleton, as illustrated in \Figure{topology}, are required to satisfy this rigidity condition.

% \newpage
\paragraph{Validity}
The calibration optimization should avoid producing degenerate configurations \todo{in our \emph{rest pose} template $\restcenters$}. For example, a pill degenerates into a sphere when one of its balls is fully contained within  the volume of the other. Analogously, a wedge can degenerate into a pill or a sphere. We monitor validity by an indicator function $\chi(\bar\ball_i)$ that evaluates to one if $\bar\ball_i$ is degenerate and zero otherwise.
% 
\todo{We make a conservative choice and use $\chi(\bar\ball_i)$, which verifies whether $\bar\ballcenter_i$ is inside $\bar\oneelement \setminus \bar\ball_i$, the element obtained by removing a vertex, as well as all its adjacent edges, from $\bar\oneelement$.}
% 
This leads to the following \todo{conditional} penalty function:
% 
\begin{equation}
E_{\text{valid}} = 
\sum_{\bar\oneelement \in \restcenters}
\sum_{\bar\ball_i \in \bar\oneelement} 
\chi(\bar\ball_i) 
\| \bar\ballcenter_i - \proj_{\bar\oneelement \setminus \bar\ball_i}(\bar\ballcenter_i) \|_2^2
\end{equation}
% 
 


%Our fitting energy $E_{\text{fit}}$ captures the alignment of the hand model with the 3D point cloud and the 2D silhouette by combining two distinct energies
%% 
%\begin{equation}
%E_{\text{fit}} =  E_{\text{3D}} + E_{\text{2D}}.
%\label{eq:fitting}
%\end{equation}
%
%
% Our data fitting performs a joint 2D-3D op- timization. Our 3D alignment ensures that every point measured by the sensor Xs is sufficiently close to the tracked model M. Si- multaneously, as we cannot create such constraints for occluded portions of the hand, we optimize for a 2D registration that ensures the tracked M lies in the sensor visual hull Ss. O
%
%
% Our priors regularize the solution to ensure the recovered pose remains likely. We determined that retaining realistic hand postures is critical, as erroneous postures can result in establishing erroneous closest-point correspondences and cause catastrophic loss of tracking.
%


\subsection*{Point cloud alignment}
%The quality of the point cloud alignment is measured using $E_{\text{3D}}$. 
The term $E_{\text{3D}}$ models a 3D geometric registration in the spirit of ICP as
%. The energy is defined as
%
\begin{equation}
    E_{\text{3D}}  = \omegacloud \sum_{\pointHtrack \in \PointsSensorHtrack} \| \pointHtrack - \proj_{\handmodel}(\pointHtrack,\parsHtrack) \|_2,
\label{eq:align3d}
\end{equation}
%
where $\|\cdot\|_2$ denotes the $\ell_2$ norm, $\pointHtrack$ represents a 3D point of $\PointsSensorHtrack$, and $\proj_{\handmodel}(\pointHtrack,\parsHtrack)$ is the projection of $\pointHtrack$ onto the hand model $\handmodel$ with hand pose $\parsHtrack$. 
%This projection defines the correspondence of $\point$ on the model. For brevity of notation, we define this correspondence as $\mathbf{y} = \proj_{\handmodel}(\point,\pars)$.
%
 Note that we compute a sum of absolute values of the registration residuals, not their squares. This corresponds to a mixed $\ell_{2}/\ell_{1}$ norm of the stacked vector of the residuals. For 3D registration such a sparsity-inducing norm has been shown to be more resilient to noisy point clouds containing a certain amount of outliers
such as the ones produced by the Creative sensor (\Figure{data}). We refer to~\cite{bouaziz_sgp13} for more details.

% \SB{not sure if we should say that here: As the 3D hand model is simply composed of cylinders, we compute the projections in close form ignoring back facing correspondences.}


\begin{figure}[b]
\centering
\flushleft
\begin{overpic} 
[width=.97\linewidth]
% [width=.97\linewidth,grid,tics=5]
{htrack/fig/occnrg/composite.pdf}
\put(100,2){\rotatebox{90}{\tiny{corresp. culling}}}
\put(100,24){\rotatebox{90}{\tiny{occlusion energy}}}
\putfilename
\end{overpic}
% \vspace{-.05in}
\caption{
% 
Correspondence computations.
The top row shows the strategy 
% originally proposed for full-body tracking by \protect\cite{ganapathi_eccv12}
used in~\protect\cite{qian_cvpr14} adapted to our gradient-based framework according to the formulation given in~\protect\cite{wei_siga12}. The bottom row shows the improved accuracy of our new approach.
% 
%The effect of replacing our view-dependent correspondence computation by an energy minimizing depth disparity~\protect\cite{wei_siga12} for portions of the model occluding the sensor data~\protect\cite{qian_cvpr14}. \MP{I don't understand the references. So the top row is Wei et al and the bottom row our method? Why refer to Qian?}
} % caption
\label{fig:occnrg}
\end{figure}

\subsection*{3D correspondences}
% The projection operators are now decoupled from the optimization and can be evaluated, rather than differentiated.
% The optimization problems in Step.1 can be solved in close form. The computation of 3D correspondences in \Equation{cp3d} can be performed either by exhaustively computing the distances to the underlying cylinder model, or by first rendering the point cloud with the same viewport of the sensor and then fetching closest points with a spatial data structure (kdtree or octree). \AT{Note that the first way of computing closest points is trivially parallelizable}
The 3D registration term involves computing the corresponding point  $\mathbf{y} = \proj_{\handmodel}(\pointHtrack,\parsHtrack)$ on the cylinder model~$\handmodel$ for each sensor point $\pointHtrack \in \PointsSensorHtrack$. 
In contrast to standard closest point search, we define the correspondence $\mathbf{y}$  as the closest point on the \emph{front-facing} part $\visiblehand$ of $\handmodel$. This includes parts of the model that are oriented towards the camera but occluded by other parts. 
 In our experiments we learned that this seemingly simple extension proved absolutely essential to obtain high-quality tracking results.
 Only considering model points that are visible from the sensor viewpoint, i.e., matching to the rendered model, is not sufficient for handling occlusions or instances of disappearing and reappearing sensor data; see \Figure{frontcorr} and \Figure{occlusion}. 
 
To calculate $\mathbf{y}$, we first compute the closest points $\pointHtrack_\mathcal{C}$ of $\pointHtrack$ to each cylinder $\mathcal{C}\in\handmodel$. Recall that our hand model consists of sphere-capped cylinders so these closest points can be computed efficiently in closed form and in parallel for each $\pointHtrack \in \PointsSensorHtrack$.
We then identify back-facing points using the dot product of the cylinder surface normal $\mathbf{n}$ at $\pointHtrack_\mathcal{C}$ and the view ray vector $\mathbf{v}$. 
%
For efficiency reasons, we use a simplified orthographic camera model where the view rays are constant, i.e., $\mathbf{v} = [0~0~1]^T$. If a point on a cylinder is back-facing ($\mathbf{n}^T\mathbf{v}>0$), we project $\pointHtrack$ onto the cylinder's silhouette contour line from the camera perspective, whose normals are orthogonal to $\mathbf{v}$.

%We first compute the orthogonal projection of $\point$ onto the cylinder's center line segment. The point on the cylinder's surface is then found by moving back along the projection axis by the cylinder radius.


%As the sensor data only contains points that are visible from the camera's point of view, we only project to the parts of the cylinder that are front-facing with respect to the camera. This includes parts of the model that are oriented towards the camera but occluded by other parts of the model. Only considering model points that are visible from the sensor viewpoint, i.e., matching to the rendered model point cloud, is not sufficient for handling occlusions or instances of disappearing and reappearing sensor data (see \Figure{frontcorr} and \Figure{occlusion}). 



% 
% Conversely, to establish closest point correspondences to $\SilhoSensor$, we first compute its distance transform in linear time using~\cite{felzenszwalb_12} (1ms on a 320x240 image).
 
A different strategy to address visibility issues has been introduced \new{ in~\cite{qian2014realtime}. These methods} propose an energy that penalizes areas of the model falling in front of the data, which is then optimized using particle swarms. This energy can be integrated into our optimization following the formulation in \cite[Eq. 15]{wei_siga12}. However, such an energy is prone to create local minima in gradient-based optimization, as illustrated in \Figure{occnrg}. Here the thumb has difficulty entering the palm region, as it must occlude palm samples before reaching its target~configuration. Our correspondence search avoids such problems.
\new{Furthermore, note how~\cite{qian2014realtime} follows a \emph{hypothesize-and-test} paradigm where visibility constraints in the form of \emph{ray-casting} are easy to include. As discussed in  \cite{ganapathi_eccv12}, such constraints are much more difficult to include in iterative optimization techniques like ours. However, our front-facing correspondences computation provides a simple and elegant way to deal with such shortcomings.}


% takes the energy formulation of~, which was optimized by particle swarm optimization, and adapts it to our gradient-based formulation; see~.

% \MP{Does this make more sense? I would still like to give an explanation of what is happening here. People might not know Qian. Can we say in one sentence what the difference is?}

%\AT{this is a dangerous statement. We implement the energy proposed in \cite{qian_cvpr14} and then linearize it. It's the gradient of this energy that is f'd, but if you follow a particle swarm strategy it does work `ok'}, highlighting the improved tracking accuracy of our approach.

\begin{figure}[t]
\centering
\begin{overpic} 
[width=\linewidth]
% [width=\linewidth,grid,tics=5]
% {fig/push/composite.pdf} %f'd up transparency
{htrack/fig/push/composite.png}
\put(44,40){\small{$\SilhoSensor$}}
\put(44,18){\small{$\SilhoSensor$}}
\put(31.5,-2){\tiny{silhouette}}
\put(54.5,-2){\tiny{w/o silhouette}}
\put(79.5,-2){\tiny{w/ silhouette}}
\putfilename
\end{overpic}
\vspace{-.1in}
\caption{
% \todo{[push]}
Our 2D silhouette registration energy is essential to avoid tracking errors for occluded parts of the hand.
When no depth data is available for certain parts of the model, a plausible pose is inferred by ensuring that the model is contained within the sensor silhouette image $\SilhoSensor$.
% correctly track occluded geometry. In these scenarios, it is the lack of data of depth measurements in certain areas that drives pose inference - by ensuring our model is behind by the sensor silhouette $\SilhoSensor$.
}
\label{fig:push}
\end{figure}

\subsection*{Silhouette alignment}
The 3D alignment energy $E_{\text{3D}}$ robustly measures the distance between every point in the 3D point cloud~$\PointsSensorHtrack$ to the tracked model $\handmodel$. However, as hands are highly articulated, significant self-occlusions are common during tracking. Such self-occlusions are challenging, because occluded parts will not be constrained when only using a 3D alignment energy. For this reason, we use a 2D silhouette term $E_{\text{2D}}$ that models the  alignment of the 2D silhouette of our rendered hand model with the 2D silhouette extracted from the sensor data as 
%
%The second term $E_{\text{2D}}$ models a 2D geometric alignment in a similar manner than the $E_{\text{3D}}$ energy as
%
\begin{equation}
    E_{\text{2D}} = \omegasilhouette \sum_{\pixelHtrack \in \SilhoRender} \| \pixelHtrack - \proj_{\SilhoSensor}(\pixelHtrack,\parsHtrack) \|_2^2,
\label{eq:align2d}
\end{equation}
%
% where $\pixel$ is a 2D point of the rendered silhouette $\SilhoRender$, and $\proj_{\SilhoSensor}(\pixel,\pars)$ denotes the projection of $\pixel$ onto $\SilhoSensor$.
where $\pixelHtrack$ is a 2D point of the \emph{rendered} silhouette $\SilhoRender$, and $\proj_{\SilhoSensor}(\pixelHtrack,\parsHtrack)$ denotes the projection of $\pixelHtrack$ onto the \emph{sensor} silhouette $\SilhoSensor$.
%over the 2D silhouette extracted from the sensor data $\SilhoSensor$. 
% \SB{not sure if we should say that here: The projections are computed using the 2D distance transform of $\SilhoSensor$~\cite{....}.}
%
\Figure{push} shows why the silhouette term is crucial to avoid erroneous poses when parts of the model are occluded. 
Without the silhouette energy, the occluded fingers can mistakenly move to wrong locations, since they are not constrained by any samples in the depth map.







%\subsection{Correspondence Search}
%
%The 3D and 2D registration energies of Equations~\ref{eq:align3d} and~\ref{eq:align2d} require the computation of point correspondences from the data to the model. Our solution improves upon existing correspondence algorithms without compromising computational efficiency.
%
%



\subsection*{2D correspondences}
In \Equation{align2d}, we compute the silhouette image $\SilhoRender$ by first rendering the hand model $\handmodel$ from the viewpoint of the sensor, caching the bone identifier and the 3D location associated with each pixel in a texture.
% for the computation of the Jacobian in \Equation{jacobian2d} are cached in a texture. 
The projection function $\proj_{\SilhoSensor}(\pixelHtrack,\parsHtrack)$ to compute the
% For each pixels in $\SilhoRender$, we then identify
 closest corresponding point to the sensor silhouette is evaluated efficiently using the 2D distance transform of $\SilhoSensor$.
 We use the linear time algorithm of~\cite{felzenszwalb_12} and store at every pixel the index to the closest correspondence.
% To perform this task efficiently, we compute the 2D distance transform of $\SilhoSensor$ with a linear time algorithm~\cite{felzenszwalb_12}, i.e., at every pixel we store the distance transform value and the index to the closest correspondence.
% simple to evaluate in close form, as they simply require us to compute the closest point to either the tracking model $\handmodel$ or the silhouette of the sensor data $\SilhoSensor$.
% At each iteration we render $\handmodel$ from the viewpoint of the sensor
% We employ the color of each pixel to uniquely identify the originating bone. This information, as well as its 3D position are necessary to compute the gradient of~\Equation{align2d}.
% Step 2 can now be approached by Gauss-Newton optimization; we perform Taylor expansion of the residuals $\mathbf{r}$ of each energy term in \Equation{icp} as $\mathbf{r} \approx \mathbf{r}_0 + \J(\mathbf{r}_0) \delta\pars$, where $\J=\partial \mathbf{r} / \partial \pars$ is the Jacobian of the residuals, and then compute the optimal update as:
% %
% \begin{eqnarray}
% \delta\pars = (\J^t \J + \alpha \mathbf{I})^{-1} \J^t (\mathbf{r} - \mathbf{r_0})
% \label{eq:gaussnewton}
% \end{eqnarray}
% %
% A description of the rows that assemble $\J$, altogether with a description of the chain-rule derivatives is given in the appendix. In the equation above, the term weighted by $\alpha$ both ensures the system remains well conditioned and stabilizes the optimization by reducing the magnitude of the computed update.



% \brief{Convergence Speed}

\begin{figure}[t]
\centering
\begin{overpic} 
[width=\linewidth]
% [width=\linewidth,grid,tics=5]
{htrack/fig/shapespace/composite.pdf}
\putfilename
\end{overpic}
\caption{An illustration of the PCA pose-space used to regularize the optimization. Black dots denote the samples of the data base. 
High likelihood poses are located nearby the mean of the latent space (dark red). 
% Note how the eigenvalues of the PCA space skew the metric in such a way that certain axis offer a broader spectrum of likely poses.
The eigenvalues of the PCA define the metric in the low-dimensional space, skewing it in certain directions. Poses that, according to this metric, are far from the mean are likely to be unnatural and will be penalized in the optimization.
}
\label{fig:shapespace}
\end{figure}

\subsection*{Wrist alignment}
\new{The inclusion of the forearm for hand tracking has been shown beneficial in~\cite{melax2013dynamics}. Our wrist alignment energy encodes a much simplified notion of the forearm in the optimization that enforces the wrist joint to be located along its axis.}
\begin{equation}
    E_\text{wrist} = \omegawrist \| \proj_\text{2D}(\mathbf{k}_0(\parsHtrack)) - \proj_{\ell}(\mathbf{k}_0(\parsHtrack)) \|_2^2,
\end{equation}
\new{
Minimizing this energy helps preventing the hand from erroneously rotating/flipping during tracking; an occurrence of this can be observed at 04:03 in the accompanying video.
% 
Here $\mathbf{k}_0$ is the 3D position of the wrist joint, and $\ell$ is the 2D line extracted by PCA of the 3D points associated with the wristband; see \Figure{wristband}. Note that $\proj_\text{2D}$ causes residuals to be minimized in screen-space, therefore the optimization of this energy will be analogous to the one of \Equation{align2d}. 
We optimize in screen space because, due to occlusion, we are only able to observe half of the wrist and this causes its axis to be shifted toward the camera.}

\begin{figure}[t]
\centering
\begin{overpic} 
[width=.8\linewidth]
% [width=.8\linewidth,grid,tics=5]
{htrack/fig/shapespaceproj/composite.pdf}
\put(36,45){$\parsHtrack$}
\put(26,31){$\tilde\parsHtrack$}
\put(26,23){$\boldsymbol{\mu}$}
\put(39,6){$\posespace$}
% \put(25,31){$\tilde\pars$}
\putfilename
\end{overpic}
\caption{
% 
An illustration of the energies involved in our pose-space prior. For illustration purposes the full dimensional parameter vector $\parsHtrack\in\mathbb{R}^3$, while latent space variable $\tilde\parsHtrack\in\mathbb{R}^2$.
% 
The PCA optimization in \protect\cite{schroder2014real} constrains the pose parameters $\parsHtrack$ to lie on the subspace~$\posespace$. Conversely, we penalize the distance of our pose from~$\posespace$~(\Equation{pcaproj}); simultaneously, we ensure our pose remains likely by preventing it from diverging from the mean of the distribution~(\Equation{pcareg}).
% 
% either limiting the freedom of motion of the hand.
% This reduces the freedom of motion, as only poses in the subspace
% This can result in erroneously recovered poses, as $\pars$ can
% 
}
\label{fig:shapespaceproj}
\end{figure}
\begin{figure}[t]
%\flushright
\begin{overpic} 
[width=.98\linewidth]
%[width=.98\linewidth,grid,tics=5]
{htrack/fig/pcaconv/composite_new.pdf}
\put(-1,38){\rotatebox{90}{\small{with pose prior}}}
\put(-1,65){\rotatebox{90}{\small{w/o pose prior}}}

\put(3.5,4){{\small{0}}}
\put(17.6,4){{\small{3}}}
\put(31.1,4){{\small{6}}}
\put(39.2,4){{\small{10}}}

\put(57.4,4){{\small{0}}}
\put(71,4){{\small{1}}}
\put(84.8,4){{\small{2}}}
\put(93.3,4){{\small{3}}}

\put(16,0){{\small{w/o data term}}}
\put(69,0){{\small{with data term}}}
\putfilename
\end{overpic}
\vspace{1em}
\caption{Beyond favoring natural poses, the data prior term also positively affects convergence speed. Top: With the same number of iterations, only with activated data term does the model fully register to the scan. The illustration below shows how the same final state requires significantly fewer iterations with the data term.
}
\label{fig:pcaconv}
\end{figure}
\subsection{Prior Energies}
\label{sec:prior}
Minimizing the fitting energies alone easily leads to unrealistic or unlikely hand poses, due to the deficiencies in the input data caused by noise, occlusions, or motion blur. We therefore regularize the registration with data-driven, kinematic, and temporal priors to ensure that the recovered hand poses are plausible. Each of these terms plays a fundamental role in the stability of our tracking algorithm, as we illustrate below.

%% and is composed of three terms
%% 
%\begin{equation}
%E_{\text{prior}} = E_{\text{pose}} + E_{\text{kinematic}} + E_{\text{temporal}}.
%\label{eq:prior}
%\end{equation}
%%


%The $E_{\text{kinematic}}$ energy aims at generating a plausible hand posture by finding a hand pose respecting some kinematic constraints, i.e., angle bounds and the avoidance of self-collisions. The $E_{\text{temporal}}$ term enforces temporal smoothness to avoid jittering and in order to predict hand parameters in  case of missing data. Finally, to achieve a tracking that produces  postures realizable by a human hand we employ a data-driven energy $E_{\text{pose}}$. 


\subsection*{Pose Space prior (data-driven)}
\begin{figure}[t]
\centering
\begin{overpic} 
[width=\linewidth]
% [width=\linewidth,grid,tics=5]
{htrack/fig/pca/composite.pdf}
\put(30,-2.5){\small{depth image}}
\put(55,-2.5){\small{without PCA}}
\put(81,-2.5){\small{with PCA}}
\putfilename
\end{overpic}
\vspace{1em}
\caption{
% 
Our pose-space regularization using a PCA prior ensures that a meaningful pose is recovered even when significant holes occur in the input data.
}
\label{fig:pca}
\end{figure}

The complex and highly coupled articulation of human hands is difficult to model directly with geometric or physical constraints. Instead, we use a publicly available database of recorded hand poses~\cite{schroder2014real} to create a data-driven prior $E_{\text{pose}}$ that encodes this coupling.
% To achieve a tracking that produces hand postures which are realizable by a human hand (e.g. we cannot bend a finger on itself), we employ a data-driven regularizer. As a simple example, it is difficult to bend the \emph{distal phalanx} without simultaneously bending the \emph{proximal phalanx}.
We construct a low-dimensional subspace of plausible poses by performing dimensionality reduction using PCA (see \Figure{shapespace}). 
% 
%\begin{eqnarray}
%E_{\text{pose}} = E_{\text{projection}} + E_{\text{mean}}.
%\end{eqnarray}
%
We enforce the hand parameters~$\parsHtrack$ to lie close to this low-dimensional linear subspace using a 
 data term
$E_{\text{pose}} = E_{\text{projection}} + E_{\text{mean}}$.
%
To define the data term, we introduce auxiliary variables $\parssub$, i.e, the PCA weights, representing the (not necessarily orthogonal) projection of the hand pose $\parsHtrack$ onto the subspace; see \Figure{shapespaceproj}.
%\MB{If $\parssub$ is the projection of $\pars$ to the PCA space, then it is \emph{not} an auxiliary variable, since it is not free to be chosen differently. It is fully defined by $\pars$ and the PCA matrix $P$ as $\parssub = P P^T (\pars-\boldsymbol{\mu}$. You seem to really use it as an auxiliary variable, but then the above statement is wrong, and I don't see why you have to add the extra variables.}
The projection energy  measures the distance between the hand parameters and the linear subspace as
% 
\begin{eqnarray}
E_{\text{projection}}  = \omegapcaproj \|(\parsHtrack - \boldsymbol{\mu}) - \proj_\posespace \parssub \|_2^2,   
\label{eq:pcaproj}
\end{eqnarray}
% 
where $\boldsymbol{\mu}$ is the PCA mean. The matrix $\proj_\posespace$, i.e., the PCA basis,
%\MB{The PCA matrix (let's call it $P$), with the PCA basis vectors in its columns, is \emph{not} a projection. The projection matrix onto the PCA subspace is $PP^T$.}
reconstructs the hand posture from the low-dimensional space. 
%\MB{Why not simply measure the distance of $\pars$ from the subspace? This should by $\| (I-PP^T)(\theta-\mu) \|$. For \Eq{pcaproj} there's no need for an extra $\parssub$. }
To avoid unlikely hand poses in the subspace, we regularize the PCA weights $\parssub$ using an energy
% 
\begin{eqnarray}
E_{\text{mean}} = \omegapcareg \|\boldsymbol{\Sigma} \parssub \|_2^2. 
\label{eq:pcareg}
\end{eqnarray}
% 
$\boldsymbol{\Sigma}$ is a diagonal matrix containing the inverse of the standard deviation of the PCA basis.
Our tracking optimization is modified to consider the pose space by introducing the auxiliary variable $\tilde\parsHtrack$ and then jointly minimizing over $\parsHtrack$ and $\tilde\parsHtrack$. \new{The difference between our approach and optimizing directly in the subspace is further discussed in~\Appendix{pca}}.
% 
\new{Note how the regularization energy in \Equation{pcareg} helps the tracking system recover from tracking failures. When no sensor constraints are imposed on the model, the optimization will attempt to push the pose toward the mean -- a statistically likely pose from which tracking recovery is highly effective.}

\Figure{pca} illustrates how the PCA data prior improves tracking by avoiding unlikely poses, in particular when the input data is incomplete.
We found that even when data  coverage is sufficient to recover the correct pose, the data term improves the convergence of the optimization as illustrated in \Figure{pcaconv}.
% 
\Figure{pcafail} shows how our regularized projective PCA formulation outperforms the direct subspace optimization proposed in previous work.




% \todo{Explain why this is better than measuring the distance to the projection.}

\subsection*{Kinematic prior}
The PCA data term is a computationally efficient way of approximating the space of plausible hand poses.
However, the PCA model alone cannot guarantee that the recovered pose is realistic. In particular, since the PCA is symmetric around the mean, fingers bending backwards beyond the physically realistic joint angle limits are not penalized by the data prior. Similarly, the PCA model is not descriptive enough to avoid self-intersections of fingers. These two aspects are addressed by the kinematic prior $E_{\text{kinematic}} = E_{\text{collision}} + E_{\text{bounds}}$.
%
%\begin{eqnarray}
%E_{\text{kinematic}} = E_{\text{collision}} + E_{\text{bounds}}.
%\label{eq:kinematic}
%\end{eqnarray}
%
Under the simplifying assumption of a cylinder model, we can define an energy $E_{\text{collision}}$ that accounts for the inter-penetration between each pair of (sphere-capped) cylinders:
% 
\begin{equation}
    E_{\text{collision}} = \omegacollision \sum_{\{i,j\}} {\chi(i,j)}(d(\cyl_i, \cyl_j) - r)^2,
\end{equation} 
%
where the function $d(\cdot,\cdot)$ measures the Euclidean distance between the cylinders axes $\cyl_i$ and $\cyl_j$, and $r$ is the sum of the cylinder radii. ${\chi(i,j)}$ is an indicator function that evaluates to one if the cylinders $i$ and $j$ are colliding, and to zero otherwise.
% \MB{I don't like the $\chi$, it messes up the equation. Why not use truncated powers, as people playing with compact kernels use: $(r-d(c_i,c_j))_+^2$, where $(x)_+^2$ is $x^2$ for $x>0$ and $0$ otherwise. It's clear what you want to model, but the $\chi$ just makes the equation (appear) unnecessarily complicated.}
The top row of \Figure{posepriors} shows how this term avoids interpenetrations of the fingers.
% \MP{Should add a figure that shows this effect} DONE
% 

\begin{figure}[t]
\centering
\begin{overpic} 
[width=\linewidth]
% [width=\linewidth,grid,tics=5]
{htrack/fig/pcafail/composite.pdf}
\renewcommand{\yoff}{-4}
\put(05,\yoff){\tiny$89\%$, \#PCA=4}
\put(30,\yoff){\tiny$96\%$, \#PCA=6}
\put(54,\yoff){\tiny$99\%$, \#PCA=9}
\put(79,\yoff){\tiny$79\%$, \#PCA=2}
\renewcommand{\yoff}{-8}
\put(08,\yoff){\tiny$\omegapcaproj=10^8$}
\put(33,\yoff){\tiny$\omegapcaproj=10^8$}
\put(57,\yoff){\tiny$\omegapcaproj=10^8$}
\put(82,\yoff){\tiny$\omegapcaproj=10^2$}
\putfilename
\end{overpic}
\vspace{.05in} % leave enough space for overpic
\caption{
%
Optimizing directly in the PCA subspace~\protect\cite{schroeder_icra14} can lead to inferior registration accuracy.
We replicate this behavior by setting $\omegapcaproj$ in Equation~\ref{eq:pcaproj} to a large value. Even when increasing the number of PCA bases to cover $99\%$ of the variance in the database, the model remains too stiff to conform well to the input. Our approach  is able to recover the correct hand pose by optimizing for projection distances even with a very limited number of bases (right).}
\label{fig:pcafail}
\end{figure}

% by optimizing for projection distance (\Equation{pcaproj}) we can
% 
% We directly compare our method to the one in  by varying the number of PCA bases. Note how are technique is capable to fit well to the data with a very limited number of PCA bases.
% 
% (bottom) To decouple improvements due to our fitting energies, we realize the subspace optimization of \protect\cite{schroeder_icra14} by setting the latent projection weight $\omegapcaproj=10^{8}$.

%computed in close form as the shortest segment between the two cylinders axes
% \SB{To prevent fingers inter-penetrating each other during tracking we instantiate collision constraints. Similarly to \cite{oiko_?}, to retain real-time performance, we only instantiate constraints between nearby finger segments.}
To prevent the hand from reaching an impossible posture by overbending the joints, we limit the joint angles of the hand model:
\begin{equation}
   E_{\text{bounds}} = \omegabounds \sum_{\theta_i \in \parsHtrack}
        \underline{\chi}(i)(\theta_i - \underline{\theta}_i)^2
            +
        \overline{\chi}(i)(\theta_i - \overline{\theta}_i)^2,
        \label{eq:bound}
\end{equation}
%
% \MB{Same suggestion here: Replace $\chi$ by a truncated power, then you don't need to introduce two new $\chi$s}
where each hand joint is associated with conservative bounds $\left[ \underline{\theta}_i,\overline{\theta}_i\right]$. For the bounds, we use the values experimentally determined by \cite{chan1995weighted}.  {$\underline{\chi}(i)$} and {$\overline{\chi}(i)$} are indicator functions. {$\underline{\chi}(i)$} evaluates to one if $\theta_i < \underline{\theta}_i$, and to zero otherwise. $\overline{\chi}(i)$ is equal to one if $\theta_i > \overline{\theta}_i$, and  zero otherwise.
% \AT{should we say something about the fact that they lose meaning unless the palm is aligned correctly? This would make}
The bottom row of \Figure{posepriors} illustrates the effect of the joint angle bounds.

\begin{figure}[t]
\flushleft
\begin{overpic} 
[width=.95\linewidth]
% [width=.95\linewidth,grid,tics=5]
{htrack/fig/posepriors/composite.pdf}
\put(100,36.5){\rotatebox{90}{\small{collision}}}
\put(100,7){\rotatebox{90}{\small{joint limits}}}
\put(9,-2){\small{color}}
\put(34,-2){\small{depth}}
\put(57,-2){\small{disabled}}
\put(82.5,-2){\small{enabled}}
\putfilename
\end{overpic}
\vspace{1em}
\caption{Kinematic priors augment the data prior to account for inconsistencies in the pose space. The collision term avoids self-collisions (top row), while the term for joint angle bounds avoids overbending of the finger joints.
% (top) Our latent pose space can contain poses with self-intersecting geometry. The collision energy prevents our tracking model from falling in these strong registration local minima. (bottom) As we only penalize the distance from the mean pose, given enough fitting constraints, our tracked model can assume unlikely poses. Joint bounds give further structure to the latent space by preventing the creation of unrealizable poses; also see \Figure{shapespaceproj}}.
}
\label{fig:posepriors}
\end{figure}



\begin{figure}[b]
\centering
\begin{overpic} 
[width=\linewidth]
% [width=\linewidth,grid,tics=5]
{htrack/fig/temporal/composite.pdf}
\put(1.2, 5){\tiny\rotatebox{90}{w/ temporal}}
\put(1.2, 25.5){\tiny\rotatebox{90}{w/o temporal}}
\put(22, 42){\tiny{50}}
\put(40, 42){\tiny{100}}
\put(59, 42){\tiny{150}}
\put(78, 42){\tiny{200}}
% labels
\put(14,-2.5){\tiny(a)}
\put(38,-2.5){\tiny(b)}
\put(62,-2.5){\tiny(c)}
\put(86,-2.5){\tiny(d)}
% plot labels
\put(42.75,69){\tiny(a)}
\put(47.4, 69){\tiny(b)}
\put(50.6, 69){\tiny(c)}
\put(55.5, 69){\tiny(d)}
% Y
\put(-.5, 56){\tiny{\rotatebox{0}{ 50}}}
\put(-.5, 68){\tiny{\rotatebox{0}{100}}}
\put(2, 72){\small{y}}
\putfilename
\end{overpic}
\caption{The effect of the temporal prior. The graph shows the trajectory of the $y$-coordinate of the  fingertip over time as the index finger is bend up and down repeatedly. The temporal prior reduces jitter, but also helps avoiding tracking artifacts that arise when fragments of data pop in and out of view. %see also \Figure{data}-(bottom).
} % caption
\label{fig:temporal}
\end{figure}
\subsection*{Temporal prior}

A common problem in particular with appearance-based methods are small-scale temporal oscillations that cause the tracked hand to jitter. A standard way to enforce temporal smoothness is to penalize the change of model parameters~$\parsHtrack$ through time, for example, by penalizing a quadratic energy accounting for velocity $\|\dot \parsHtrack\|^2$ and acceleration $\|\ddot \parsHtrack\|^2$~\cite{wei_siga12}. 
% 
% \MP{Please verify that it's ok to cite them here} \AT{it's correct}
% 
However, if we consider a perturbation of the same magnitude, it would have a much greater effect if applied at the root, e.g., global rotation, than if applied to an element further down the kinematic tree, e.g., the last phalanx of a finger. 
%\AT{we should say that this was what was done in ?} \MP{What did they do, our method, or the standard approach described above?}
%\AT{they do the standard approach. Never seen what we do around...}
% \todo{Also, it is not intuitive to understand how to tune its weighting parameter... compare angles with translations, and then to other energies... little meaning!!}
Therefore, we propose a solution that measures the velocity and acceleration of a set of points attached to the kinematic chain. We consider the motion of vertices $\skelpoint$ of the kinematic chain $\mathcal{K}$ (\Figure{handmodel}) and build an energy penalizing the velocity and acceleration of these points:
% \MB{I would call the vertices $\skelpoint_i$ to match the figure}
\begin{eqnarray}
    E_{\text{temporal}} = \omegatemporalst \sum_{\skelpoint_i \in \mathcal{K}} \| \dot \skelpoint(\parsHtrack) \|_2^2 + \omegatemporalnd \sum_{\skelpoint_i \in \mathcal{K}} \| \ddot \skelpoint(\parsHtrack) \|_2^2.
\end{eqnarray}
%
% \MB{Mention how you compute the time-derivatives.}
\Figure{temporal} illustrates how the temporal prior reduces jitter and improves the overall robustness of the tracking; see also accompanying video.

%!TEX root = ../../paper.tex
\begin{teaserfigure}
\begin{overpic} 
[width=\linewidth]
% [width=\linewidth,grid,tics=10]
{fig/teaser/item.pdf}
\put(2.2,9){$\color[RGB]{198,94,125}  \text{diag}({\star\Sigma}^{-1})$}
\put(2.2,4){$\color[RGB]{103,177,159} \text{diag}({\hat\Sigma}^{-1})$}
\myfigurename{}
\end{overpic}
\centering
\vspace{-.2in}
\caption{
% 
% 
% 
Our adaptive hand tracking algorithm optimizes for a tracking model on the fly, leading to progressive improvements in tracking accuracy over time.   
{\bf Above}: hand surface color-coded to visualize the spatially-varying confidence of the estimated geometry. Insets: color-coded \emph{cumulative} certainty.  Notice how in the last frame all parameters are certain. 
{\bf Below}: histograms visualize the certainty of each degree of freedom, that is, the diagonal entries of the inverse of the covariance estimate from:
(a) data in the current frame $\star\Sigma$, or 
(b) the information $\hat\Sigma$ accumulated through time by our system.
% These results are best appreciated by watching the supplementary \textbf{video}.
%
%
% 
}
\label{fig:teaser}
\end{teaserfigure}
